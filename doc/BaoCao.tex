\documentclass[12pt]{article}

\usepackage[utf8]{vietnam}
\usepackage{indentfirst}
\usepackage{url}
\usepackage{listings}

\lstset{language=sh,basicstyle=\ttfamily,breaklines=true}

\begin{document}

\title{BÁO CÁO ĐỒ ÁN 2}
\author{Nguyễn Trần Hậu - MSSV 1612180
\and Nguyễn Chí Thức - MSSV 1612677}
\date{1/12/2018}
\maketitle

\begin{abstract}
Hướng dẫn build syscall và hook trong Linux.
Nhóm em sử dụng distro Debian bản 8 dành cho 32bit với kernel bản 3.16.61 để thực hiện đồ án này.
\end{abstract}



\section{Syscall}
Để tạo một syscall trong hệ điều hành Linux: \\
$\rightarrow$ Cần phải lấy source của kernel \\
$\rightarrow$ Thêm syscal vào source \\
$\rightarrow$ Build kernel từ source vừa chỉnh sửa \\
$\rightarrow$ Boot distro Linux bằng kernel vừa build. \\

\subsection{Chuẩn bị source}

Trong thư mục nộp bài có thư mục \textit{src}.
Trong thư mục \textit{src}, có 3 thư mục con là \textit{hook}, \textit{syscall}, và \textit{test\_syscall} \\

Down kernel source về, rồi giải nén vào \textit{/urs/src/}
\begin{lstlisting}
wget https://cdn.kernel.org/pub/linux/kernel/v3.x/linux-3.16.61.tar.xz
sudo tar -xvf linux-3.16.61.tar.xz -C /usr/src/
\end{lstlisting}

Copy thư mục \textit{pidtoname} và \textit{pnametoid} trong thư mục \textit{syscall} vào thư mục kernel source vừa giải nén.
Đây là 2 syscall để thêm vào kernel. \\


Thêm vào cuối dòng \textit{core-y} trong file \textit{Makefile} của thư mục kernel source tên của 2 syscall như sau
\begin{lstlisting}
core-y += kernel/ mm/ fs/ ipc/ security/ crypto/ block/ pnametoid/ pidtoname/
\end{lstlisting}


Vào thư mục \textit{arch/x86/syscalls/} của thư mục kernel source
và thêm vào cuối file \textit{syscall\_32.tbl}
\begin{lstlisting}
xxx i386 pnametoid sys_pnametoid
yyy i386 pidtoname sys_pidtoname
\end{lstlisting}
lấy số của dòng cuối cùng trong file \textit{syscall\_32.tbl}, cộng 1 ra xxx, cộng tiếp cho 1 ra yyy

Vào thư mục \textit{include/linux/} của thư mục kernel source
và thêm vào cuối file \textit{syscalls.h}
\begin{lstlisting}
asmlinkage int sys_pnametoid(char *name);
asmlinkage int sys_pidtoname(int pid, char *buf, int len);
\end{lstlisting}

\subsection{Build và cài đặt kernel}
Quay lại thư mục kernel source, build kernel
\begin{lstlisting}
sudo make menuconfig
sudo make
\end{lstlisting}

Sau khi build thành công, cài đặt kernel mới rồi khởi động lại máy
\begin{lstlisting}
sudo make modules_install install
sudo reboot
\end{lstlisting}

\subsection{Test}
Vào thư mục \textit{test\_syscall} trong thư mục \textit{src}, chạy test
\begin{lstlisting}
make test_pnametoid
make test_pidtoname
\end{lstlisting}

\section{Hook}
Hook là một kernel module, thay đổi syscall của hệ điều hành bằng syscall định nghĩa sẵn trong hook. \\
Trong hook module: \\
Lúc \textit{init\_module}, thay syscall cũ bằng syscall mới trong syscall table \\
Lúc \textit{exit\_moduel}, thay syscall mới bằng syscall cũ trong syscall table \\

Vào thư mục \textit{hook} trong thư mục \textit{src}, build hook
\begin{lstlisting}
make
\end{lstlisting}

Sau khi build hook thành công, cài đặt hook là kernel module vào hệ điều hành
\begin{lstlisting}
sudo insmod hook_module.ko
\end{lstlisting}

Để kiểm tra hook hoạt động, vào \textit{dmesg} để đọc log.

\begin{thebibliography}{9}

\bibitem{t1} Basics of Making a Rootkit: From syscall to hook! \\
\url{https://uwnthesis.wordpress.com/2016/12/26/basics-of-making-a-rootkit-from-syscall-to-hook/}

\bibitem{t2} Linux kernel articles \\
\url{https://blog.guillaume-gomez.fr/Linux-kernel/1/1}

\end{thebibliography}

\end{document}